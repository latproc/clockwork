\input cwebmac
% Note that to use the listing command to read the source
% file into an appendix, you need to enable eplain
% which will clash with cwebmac because of the multiple use of ifpdf
% the easiest way around this is to input cwebmac after eplain and
% remove the auto generated input cwebmac at the beginning
% of {thisfile}.tex
%\input eplain
%\input cwebmac

\acrofalse
\def\title{TITLE}
%\pagewidth=15.5cm
%\pageheight=25.5cm
%\fullpageheight=25.8cm
%\setpage
\input graphicx
\input picture
\def\startlist{\medskip\leftskip=2pc}
\def\endlist{\medskip\leftskip=0pt}
\def\label#1{\expandafter\let\csname lbl#1\endcsname\secno}
\def\ref#1{\csname lbl#1\endcsname}
	
\noinx
\nosecs
\nocon
\noatl
%
%\titletrue
%\vskip1in\noindent{\titlefont \title}\hfill\par
%\vskip0.5cm\noindent{Author\par\noindent\today}\hfill\par



\N{1}{1}Introduction.\medskip

\noindent
This program attempts to control a conveyor carrying a wool bale
by appropriate variations of a an power output value
given an input reading of a position encoder. There
are several practical considerations such as the fact that:

\startlist
$\bullet$ the conveyor control hardware has a certain delay before
control instructions have an effect at the output

$\bullet$ the control change itself takes time to move from
one setting to another, this time may be insignificant

\endlist

The program attempts to determine an approximate power setting
to achieve a certain speed and makes small adjustments to that level
to control the speed.

The program outline is constructed as follows.
% note that the 'h' tag inserts defines created using the 'd' tag

\Y\B\X3:Include necessary header files\X\6
\ATH\6
\X2:Define fundamental types and declarations needed by other declarations\X\6
\X4:Declare types, functions and shared variables\X\6
\X5:Implement functions\X\7
\&{int} \\{main}(\&{int} \\{argc}${},\39{}$\&{char} ${}{*}\\{argv}[\,]){}$\1\1%
\2\2\6
${}\{{}$\1\6
\X6:Declare main function variables\X\6
\X7:Initialise the application structures\X\6
\X8:Perform the application task\X\6
\&{return} \T{0};\6
\4${}\}{}$\2\par
\fi

\M{2}The \CEE/ language does not have a boolean type so we
define one for use throughout our program.
%\Y\B\F\.{BOOL} \5
\\{double}\par
\B\4\D$\.{FALSE}$ \5
\T{0}\par
\B\4\D$\.{TRUE}$ \5
\T{1}\par
\Y\B\4\X2:Define fundamental types and declarations needed by other
declarations\X${}\E{}$\6
\&{typedef} \&{int} \&{BOOL};\par
\U1.\fi

\M{3}The program needs some standard libraries for access to
input and output on standard io streams and for access to
functions such as {\tt exit()}.

\Y\B\4\X3:Include necessary header files\X${}\E{}$\6
\8\#\&{include} \.{<stdio.h>}\6
\8\#\&{include} \.{<stdlib.h>}\par
\U1.\fi

\M{4}In this template there are no functions or shared variables.

\Y\B\4\X4:Declare types, functions and shared variables\X${}\E{}$\par
\U1.\fi

\M{5}There are no functions to implement.

\Y\B\4\X5:Implement functions\X${}\E{}$\par
\U1.\fi

\M{6}There are no variables.

\Y\B\4\X6:Declare main function variables\X${}\E{}$\par
\U1.\fi

\M{7}No initialisation is required.

\Y\B\4\X7:Initialise the application structures\X${}\E{}$\par
\U1.\fi

\M{8}This is a hello world application so we simply
say hello.

\Y\B\4\X8:Perform the application task\X${}\E{}$\6
\\{printf}(\.{"Hello\ World\\n"});\par
\U1.\fi

\N{1}{9}Programmers notes.
Under Mac~OS~X or Linux, the program
can be compiled and linked using:
\startlist\par {\tt ctangle controller.w}
\par {\tt cc -o controller controller.c}
\endlist

\fi

\M{10}Index.\medskip

% include the C source (requires eplain)
% * The \CEE/ source file.
%
%\listing{controller.c}

\fi


\inx
\fin
\con
