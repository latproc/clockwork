%% LyX 2.0.5.1 created this file.  For more info, see http://www.lyx.org/.
%% Do not edit unless you really know what you are doing.
\documentclass[english]{article}
\usepackage[T1]{fontenc}
\usepackage[latin9]{inputenc}
\usepackage[a4paper]{geometry}
\geometry{verbose,tmargin=1cm,bmargin=2cm,lmargin=1cm,rmargin=1cm,headheight=1cm,headsep=1cm,footskip=1cm}
\usepackage{graphicx}

\makeatletter

%%%%%%%%%%%%%%%%%%%%%%%%%%%%%% LyX specific LaTeX commands.
%% A simple dot to overcome graphicx limitations
\newcommand{\lyxdot}{.}


%%%%%%%%%%%%%%%%%%%%%%%%%%%%%% Textclass specific LaTeX commands.
\newenvironment{lyxcode}
{\par\begin{list}{}{
\setlength{\rightmargin}{\leftmargin}
\setlength{\listparindent}{0pt}% needed for AMS classes
\raggedright
\setlength{\itemsep}{0pt}
\setlength{\parsep}{0pt}
\normalfont\ttfamily}%
 \item[]}
{\end{list}}

\makeatother

\usepackage{babel}
\begin{document}

\title{Communications}
\maketitle
\begin{abstract}
This document describes the future communications framework for Clockwork.
A Clockwork driver program provides a runtime environment to execute
several machines in parallel. Each driver may communicate with other
programs via message passing and publisher/subscriber models. 
\end{abstract}

\section{Background}

The Clockwork interpreters have several communication interfaces to
support various control protocols, web server display, system monitoring
and to provide a commandline interface. These interfaces have been
introduced on the basis of project requirements and are inconsistent
in design and implementation. This document addresses the various
communications interfaces and describes legacy and future implementations
for
\begin{itemize}
\item EtherCAT
\item Modbus
\item commandline (iosh)
\item Internet of Things
\item web clients
\end{itemize}

\section{Architecture}

The latproc system is to be split into a communications hub with scripting
engines and communications modules all interconnected by a shared
memory model and messaging system. In the first instance, Redis and
�MQ provide the shared memory and messaging infrastructure. The shared
repository provides access to shared static data, it may be implemented
in a variety of database/memory systems but the key observation is
that the repository is not expected to notify Clockwork instances
of data changes.

\begin{figure}
\begin{centering}
\includegraphics{\lyxdot \lyxdot /\lyxdot \lyxdot /\lyxdot \lyxdot /\lyxdot \lyxdot /doc/communications_images/architecture1}
\par\end{centering}

\caption{Clockwork messages and data sharing}


\end{figure}


On startup, a Clockwork driver
\begin{itemize}
\item interrogates a repository for 

\begin{itemize}
\item machine definitions 
\item interface definitions
\end{itemize}
\item creates shadow machines for machines that are updated externally
\item establishes connections as described in the interface definitions
\item begins execution of the machines
\end{itemize}
Note that failure to establish communications does not cause an instance
to fail its startup; it will continue to establish the connection
until aborted by an external source. Communication interfaces can
be interrogated within Clockwork and the channel state can be used
to affect the execution of the models.


\section{Interprogram communication}


\subsection{Interface Definitions}

An interface between Clockwork drivers is introduced via the CHANNEL
object:
\begin{lyxcode}
channel\_name~CHANNEL(host:hostname,~port:portname)~\{
\begin{lyxcode}
IDENTIFIER~id\_string~';'

IDENT~short\_id\_string~';'~\#~short~form~of~above~id

KEY~key\_string~';'~\#~a~preshared~key~used~in~authentication

VERSION~version\_sequence~';'

SHARES~block\_name,~\ldots{}~';'

MONITORS~machine\_name~{[}~interface~{]},~\ldots{}~';'

UPDATES~machine\_name~{[}~interface~{]},~\ldots{}~;

SENDS~message\_name,~\ldots{}~';'

RECEIVES~command\_name,~\ldots{}~';'
\end{lyxcode}
\}
\end{lyxcode}
A channel may specify a set of messages sent and commands received
on the interface but there is no way to define the parameters of those
messages or commands (TBD). The clauses are described as follows:
\begin{description}
\item [{IDENTIFIER}] an internal name for the channel, primarily used to
validate that connections are talking to the channel they expect to,
if this is not provided, the channel name is used.
\item [{IDENT}] is provided is the shortform identifier, if not given,
the first six characters of the IDENTIFIER are used and if that is
not available, the first six characters of the channel name (right-padded
with spaces). The IDENT is primarily used for filtering as all �MQ
messages begin with the IDENT string.
\item [{KEY}] a pre-shared key. If the channel definition includes an IDENTIFIER
option, both sides must be able to demonstrate that they know the
identifier. The identifier itself should never be transmitted on the
channel.
\item [{VERSION}] a numeric value that provides the version of the interface
\item [{SHARES}] lists the memory blocks that are shared between the client
and the server. Each memory block must be known to both the server
and client
\item [{MONITORS}] lists the machines that the server will send updates
for. Each listed machine may also include an interface definition
that further describes the states and properties of the machine. Wildcards
may be used to define the machines that can be monitored at the receiver.
The client may request a reduced set of machines. If no MONITORS clause
is given property and state changes will not be forwarded to the channel.
\item [{UPDATES}] lists the machines that the client will send updates
for. Each listed machine may also include an interface definition
that further describes the states and properties of the machine.
\item [{SENDS}] lists the messages that will be accepted by the server. 
\item [{RECEIVES}] lists the messages that will be accepted by the client.
\end{description}
Clauses in a channel definition are all optional.

By way of an example, a BIT has the interface:
\begin{lyxcode}
BIT~INTERFACE~\{~
\begin{lyxcode}
on~STATE;~off~INITIAL;~

COMMAND~toggle,turnOn,turnOff;
\end{lyxcode}
\}
\end{lyxcode}
A channel definition is directional, how the clauses apply to the
machine depends on where the machines are instantiated. A channel
may also include templates that define the machines that are monitored
or updated via the channel. The definition may describe the states
to be exchanged, a state that is the startup or default state and
the properties that are to be exchanged on the interface.
\begin{lyxcode}
machine\_class~INTERFACE~'\{'~
\begin{lyxcode}
state1~STATE~';'

state2~STATE~';'

initial\_state~INITIAL~';'

STATES~'\{'~state3,~state4,~\ldots{}~'\}'

property1~PROPERTY~{[}~READONLY~|~READWRITE~{]}~';'

PROPERTIES~'\{'~property2~{[}~READONLY~|~READWRITE~{]},~\ldots{}~'\}'
\end{lyxcode}
'\}'
\end{lyxcode}
If an interface definition is provided and includes any state definitions,
only the states defined will be exchanged. Similarly if properties
are included in the definition, only changes to the named properties
will be exchanged.

For example, in the following definition, only the states `on' and
`off' of the switch are communicated to the connecting machine.
\begin{lyxcode}
switch~INTERFACE~\{~STATES~\{~on,~off~\}~\}
\end{lyxcode}
If the switch enters another state (eg., disabled, no message is sent
to the listener and the switch is assumed to be still in the last
state communicated. (TBD)


\subsection{Shared Memory and sampling}

Clockwork is fundamentally a program that looks at changes in a memory
map and uses those changes to affect changes in a state model of various
simulated machines. It should be possible to share any memory block
with clockwork, no matter the source. A block may represent a device
or it may be divided into regions that relate to the device in some
way, for example in with Beckhoff EtherCAT(tm) the `Module' refers
to a physical device, in MODBUS, \emph{coils} and \emph{registers}
refer to distinct memory regions.

Each block may have properties that descibe attributes common to all
regions in the block 
\begin{lyxcode}
ModbusPanel~BLOCK~\{
\begin{lyxcode}
X~BIT{[}256{]}~IN;

Y~BIT{[}256{]}~OUT;

V~WORD{[}256{]}~IN~OUT;
\end{lyxcode}
\}

panel~ModusPanel~
\end{lyxcode}
Given the above definition, Clockwork may bind fields of the block
to a machine, for example:
\begin{lyxcode}
start\_button~panel~X1;
\end{lyxcode}
defines \emph{start\_button} as a single bit object in clockwork,
linked to coil X01 in the modbus memory block. 


\subsection{Issues}
\begin{enumerate}
\item How do we manage the startup conditions? When a client connects? When
the server restarts?
\item How large is a beckhoff master memory block?
\item How quickly can we send master block updates to a clockwork program.
What latency can we expect?
\item Properties that are updated by external machines via the PROPERTY
command should not activate the property update until the 
\item Clockwork drivers need a parameter to set them into insecure mode
to allow clients to define the connection properties.
\item If a message send fails there is no way for the clockwork program
to know
\item When a message arrives it should not be delivered unless the target
machine has no active actions
\end{enumerate}

\subsection{Message Format}

Communication between all components is via a structured message using
the JSON format.
\begin{lyxcode}
command~string,

need\_response~bool~/{*}~is~this~used/needed?~TBD~{*}/

parameters~array~{[}
\begin{lyxcode}
item~\{
\begin{lyxcode}
type~B~|~BOOL~|~I~|~INTEGER~|~U~|~UNSIGNED~|~S~|~STRING~|~F~|~FLOAT~|~N~|~NAME

value~<depends~on~type>
\end{lyxcode}
\}
\end{lyxcode}
{]}
\end{lyxcode}
Note that names conform to clockwork naming rules: ALPHA {[} ALPHA
NUMBER \_ {]} ...


\subsection{Dynamic configuration}

A channel specification needs to be read by both ends of the channel.
To facilitate the use of runtime sampling systems, a channel specification
can be sent by the client when the connection is made. The channel
specification may include interface specifications (TBD). 


\subsection{Protocol}

When a client connects to a server it is required to initiate a negotiation
of the channel configuration for the connection. This can be done
by a command message with the command: CHANNEL and the name of the
channel. In this case, the server is expected to refer to its repository
for a matching channel (and version\ldots{}). 

The communications protocol is not secure; it has a simple authentication
process that will be upgraded in the future.
\begin{itemize}
\item The connecting party identifies the channel to be used:

\begin{lyxcode}
CHANNEL~channel\_name~';'
\end{lyxcode}

this message is encoded using JSON:
\begin{lyxcode}
\{\textquotedbl{}command\textquotedbl{}:\textquotedbl{}CHANNEL\textquotedbl{},\textquotedbl{}params\textquotedbl{}:{[}\textquotedbl{}name\textquotedbl{},\textquotedbl{}myChannel\textquotedbl{}{]}\}
\end{lyxcode}
\item If the server does not have a matching configuration, the client may
send the configuration (TBD do we want to allow this? Probably require
that cw be run in an insecure mode).
\item If a channel cannot be agreed, the connection will be terminated. 
\item If an identifier is provided, it will be used as a pre-shared key
to authenticate the connection. The server will send a keyword challenge
to the client and the client will be expected to combine this with
the secret identifier for the channel and send back an encrypted value.
The client will also authenticate the server in the same way.
\item If a version number is provided, both ends may emit a warning or disconnect,
depending on other options that may be in force.
\item The connecting party 
\end{itemize}

\section{Examples}


\subsection{Weight sampling}

Consider a weight sampling machine that is implemented with a remote
(client) computer that communicates with the device and a master (server)
computer that uses the sampled weight as part of its machine management. 


\subsubsection{Shared configuration}

We define a channel for communication between the computers and define
an interface that outlines the interesting states and properties of
the device interface. Both the client and server:
\begin{quotation}
ScalesChannel CHANNEL scales \{
\begin{quotation}
MONITORS scales ScalesInterface;
\end{quotation}
\}

ScalesInterface INTERFACE \{
\begin{quotation}
PROPERTIES weight, status;

stable STATE;

unstable INITIAL;
\end{quotation}
\}
\end{quotation}

\subsubsection{Server configuration}

Using this information, the master computer can define a machine that
represents a remote scales machine conforming to the ScalesInterface
and use this to implement some logic. The monitor uses the state of
the remote system to determine whether the weight is valid but also
demonstrates the use of a timer to `debounce' the remote state change:
\begin{lyxcode}
ScalesMachine~ScalesInterface;

WeightMonitor~MACHINE~scales~\{
\begin{lyxcode}
waiting~WHEN~scales~IS~unstable;

stablising~WHEN~scales~IS~stable~AND~TIMER~<~100;

overweight~WHEN~scales~IS~stable~AND~scales.weight~>=~100;

ready~WHEN~scales~IS~stable~AND~scales.weight~>=~10;

underweight~WHEN~scales~IS~stable;
\end{lyxcode}
\}

remote\_scales~ScalesMachine;

monitor~WeightMonitor~remote\_scales;
\end{lyxcode}
The host machine executes the above without needing any additional
implementation details for the remote device but it needs to link
the remote scales to the channel by instantiating the channel:
\begin{lyxcode}
scales\_channel~ScalesChannel(port:9999)~remote\_scales;
\end{lyxcode}

\subsubsection{Client configuration}

The Clockwork driver that implements the scales defines a machine
to interpret data from the hardware. Device connector will be used
in this case to prepare the properties of this machine (not shown
here).
\begin{lyxcode}
SCALES~MACHINE~\{
\begin{lyxcode}
OPTION~weight~0;

OPTION~status~``x'';

unstable~WHEN~status~==~``U'';

stable~WHEN~status~==~``S'';

unknown~DEFAULT;
\end{lyxcode}
\}

scales\_machine~SCALES;

scales\_channel~ScalesChannel(host:''master'',~port:9999)~scales\_machine;
\end{lyxcode}
When the client driver starts, it attempts to instantiate the scales
channel by making a connection to the host and port provided. The
scales channel enters a 'connected' state once communication is established.
In the situation where both sides of the channel implement the scales
the channel enters an 'error' state.


\subsection{System Monitoring}

To setup a system monitor, a pattern can be used so that a range of
machines can be selected without the need to name each machine. Consider
an application that monitors a Clockwork instance to record state
changes of a set of machines.
\begin{lyxcode}
MonitorChannel~CHANNEL~\{
\begin{lyxcode}
MONITORS~`{*}`;
\end{lyxcode}
\}
\end{lyxcode}
Note that no parameters are passed to the channel and a wildcard is
used to specify the machines to be monitored. The client instantiates
the MonitorChannel and when the connection is established, the server
transmits the current state and properties of all matching machines.

A wildcard can be associated with an interface, giving greater control
of what is monitored:
\begin{lyxcode}
MonitorChannel~CHANNEL~\{
\begin{lyxcode}
OPTION~Initialisation~FALSE;

MONITORS~`{*}`~MonitorInterface;
\end{lyxcode}
\}

MonitorInterface~INTERFACE~\{
\begin{lyxcode}
PROPERTIES~`\textasciicircum{}{[}A-Z{]}.{*}`;
\end{lyxcode}
\}
\end{lyxcode}
In the above example, the monitor interface specifies that only properties
with names that begin wtih a capital letter will be sent. A similar
specification can be used for transmitted states.


\section{Implementation}

At the client end, clockwork uses the channel object to establish
and maintain a server connection. When the connection is setup, the
current state of the machines in the 'UPDATES' list are sent and the
server is expected to also send the state of all the machines in the
'MONITORS' list. When a connection breaks, the client sets the state
of all 'MONITORS' machines to their initial/default state.

At the server end, clockwork instantiates a machine for each entry
in the 'UPDATES' list and updates the state of this machine each time
a message is received. This local instantiation shadows the state
of the remote machine and can be used by other local machines as normal.


\subsection{Implementation notes}
\begin{itemize}
\item A MachineInterface object now subclasses MachineInstance
\item A MachineInterface is a representative or `proxy' for a MachineInstance
\item MachineInterface objects are distinguished when setting up a channel
and are not normally used as a source.
\item MachineInterface objects can be used to propagate state to other instances
of clockwork; this is done by registering a distance (hops) to the
actual MachineInstance that they represent
\item There are issues with setup and teardown of connections and how this
effects messaging. TBD
\end{itemize}

\section{Predefined Channels}


\subsection{EtherCAT}

When the EtherCAT driver connects to Clockwork, it announces the modules
that it has control over either by providing the names from the shared
repository or by providing the definitions directly on the channel.
\begin{lyxcode}
Module~BLOCK~\{
\begin{lyxcode}
OPTION~position~1;

OPTION~alias~0;

pdo1~REGION~\{
\begin{lyxcode}
OPTION~SM~3;

OPTION~index~0x1a00;

OPTION~name~``Inputs'';

BIT{[}8{]};
\end{lyxcode}
\}
\end{lyxcode}
\}
\end{lyxcode}
A Clockwork script can bind objects to fields:
\begin{lyxcode}
EL1814~MODULE~1,~0;~\#~bit~0~of~module~with~position~1
\end{lyxcode}
The above legacy syntax is not general enough (TBD)

The EtherCAT polling program communicates with clockwork via two message
queues. Each message contains a mask and a bit set. The receiver of
the message updates the bits in processing memory that match the mask
and ignores other bits.

In clockwork, when IO is updated, a bits are set in the mask due to
be sent to the ethercat interface and the value is updated. When all
components are updated in process memory, the mask and the data are
sent to the EtherCAT interface. When the EtherCAT thread starts it
receives an IO mask from clockwork that defines all of the bits that
clockwork is interested in. If there are no relevant changes in the
bitmap the thread does not send anything to clockwork. When clockwork
updates output points, it messages the updates to the ethercat thread.

TBD
\begin{enumerate}
\item handle queue growth - clear the queue up to the last entry in both
directions. do we care if clockwork toggles an output faster than
the io can detect it?
\item in reading data changes, the queue should be 'played' through clockwork
in order to ensure no transitions are lost.
\end{enumerate}
Default values

When a driver starts it needs to be provided with default values for
all analogue values. These defaults are loaded whenever there is a
problem with communication to clockwork. 


\subsection{Modbus}

Clockwork provides a facility to export properties, states and commands
of machines to modbus. The current syntax provides for automatically
generated addresses but is too tightly coupled to modbus. A more general
approach is needed (TBD)


\subsection{Persistence}


\section{Legacy Channels}

This section is partial documentation of previous channel implementations


\subsection{Monitor channel}

This is the main channel by which state changes are published, default
port 5557

message structure:

command := PROPERTY | STATE

need\_response := FALSE

parameters := {[} \{ N, <name> \}, \{ <type>, <value> \} {]}


\subsection{Modbus channel}

This channel is used to communicate with the modbus daemon, that daemon
communicates to devices with the modbus protocol

message structure:

The following commands may be sent to the modbus daemon
\begin{description}
\item [{STARTUP}] tells modbus to load the configuration and obtain initial
values for all coils etc
\item [{DEBUG~ON}] tells modbusd to emit debug information
\item [{DEBUG~OFF}] tells modbusd to stop emitting debug information
\item [{UPDATE}]~
\end{description}

\subsection{Persistence channel}

This is used to report changes of property values and state in persistent
machines

Message structure
\begin{lyxcode}
command~string,

parameters~array~{[}
\begin{lyxcode}
item~\{
\begin{lyxcode}
type~B~|~BOOL~|~I~|~INTEGER~|~U~|~UNSIGNED~|~S~|~STRING~|~F~|~FLOAT~|~N~|~NAME

value~<depends~on~type>
\end{lyxcode}
\}

...
\end{lyxcode}
{]}
\end{lyxcode}

\subsection{Device connector}

There is two way communication with device connector sending property
commands and receiving property or state commands or messages. 
\begin{description}
\item [{Messages}] Device connector may simply read values and pass them
to clockwork with no guarantee of delivery, these messages may be
lost if something goes wrong at either end but for repetitively sampled
data, this is not necessarirly a problem. 
\item [{Requests}] are sent from clockwork to device connector sometime
to be handled by device connector and sometimes to pass through to
the device. The requests are to be formated as the device requires
them and a reply is to be sent to clockwork, either from the device
or from device connector. Since device connector does not know anything
about the device, the request provides additional information.
\item [{Replies}] are sent from device connector to clockwork. If the request
included a boolean option ``ack\_needed'' (true), device connector
simply sends ``OK'' when the request has been forwarded to the device.
Otherwise, device connector forwards the request to the device and
collects its response, using a pattern supplied in the request.
\item [{Commands}] are sent from device connector to clockwork. Commands
require an acknowledgment and sometimes require data. Device connector
can be configured to supply data matching a pattern via the PROPERTY
command or STATE command. Subexpressions in the pattern will become
additional parameters.
\end{description}

\subsubsection{Messages}

For data streaming, matched sub expressions are passed as parameters.
?? incomplete

Message structure:
\begin{lyxcode}
DATA~string

list\_name~name,~

params~array~{[}~value,~...{]}
\end{lyxcode}

\subsubsection{Requests}
\begin{lyxcode}
REQUEST~string

id~integer

params~array~{[}~value,~...~{]}
\end{lyxcode}

\subsubsection{Replies}


\subsubsection{Commands}


\section{Messaging system}

The messaging system attempts to deal with various interprocess issues,
using zmq for the heavy lifting but also providing application features:
\begin{itemize}
\item Clients negotiating a private communication channel
\item Dealing with replies
\end{itemize}
All applications include the notion that some data is transitory and
some is not, transitory data is sent without regard to whether anyone
is listening, this is used for repeated messages such as sensor data
streaming. For non-transitory data, a request/response method is used.


\section{Performance}

This may need to be optimised, we could use a temlpate system; a property
message looks as follows, where the user data is given in \textbackslash{}\$1,
\textbackslash{}\$2 etc. Note that the value (\textbackslash{}\$3)
item needs to be converted into a JSON string with the appropriate
escape charaters but apart from that the template can be filled fairly
quickly.
\begin{lyxcode}
\{~\textquotedbl{}command\textquotedbl{}:~\textquotedbl{}PROPERTY\textquotedbl{},~\textquotedbl{}params\textquotedbl{}:~

{[}

\{~\textquotedbl{}type\textquotedbl{}:~\textquotedbl{}NAME\textquotedbl{},~\textquotedbl{}value\textquotedbl{}:~\$1~\},

\{~\textquotedbl{}type\textquotedbl{}:~\textquotedbl{}NAME\textquotedbl{},~\textquotedbl{}value\textquotedbl{}:~\$2~\},

\{~\textquotedbl{}type\textquotedbl{}:~TYPEOF(\$3),~\textquotedbl{}value\textquotedbl{}:~\$3~\}

{]}~\}\end{lyxcode}

\end{document}
