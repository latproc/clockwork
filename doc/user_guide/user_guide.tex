%% LyX 2.0.5.1 created this file.  For more info, see http://www.lyx.org/.
%% Do not edit unless you really know what you are doing.
\documentclass[english]{article}
\usepackage[T1]{fontenc}
\usepackage[latin9]{inputenc}
\usepackage{listings}
\usepackage[a4paper]{geometry}
\geometry{verbose,tmargin=1cm,bmargin=2cm,lmargin=1cm,rmargin=1cm,headheight=1cm,headsep=1cm,footskip=1cm}

\makeatletter
%%%%%%%%%%%%%%%%%%%%%%%%%%%%%% Textclass specific LaTeX commands.
\newenvironment{lyxcode}
{\par\begin{list}{}{
\setlength{\rightmargin}{\leftmargin}
\setlength{\listparindent}{0pt}% needed for AMS classes
\raggedright
\setlength{\itemsep}{0pt}
\setlength{\parsep}{0pt}
\normalfont\ttfamily}%
 \item[]}
{\end{list}}

\makeatother

\usepackage{babel}
\begin{document}

\title{Latproc Tools User Guide}

\maketitle

\section{Preface}

Please note that this is a very early draft.


\section{Introduction}

This guide describes the tools that come with the Latproc project
(https://github.com/latproc); the programs include:
\begin{description}
\item [{beckhoffd}] a program to provide a zmq interface to an EtherCAT
installation. \emph{Not currently working}
\item [{cw}] a daemon that runs clockwork programs
\item [{device\_connector}] a program to interface between iod and external
programs
\item [{iod}] a daemon that talks to EtherCAT to interact with io hardware
using the clockwork language
\item [{iosh}] a shell to interact with iod
\item [{modbusd}] an interface between iod and modbus
\item [{persistd}] a basic persistence daemon to record state changes from
iod 
\item [{zmq\_monitor}] a program to monitor zmq messages published by iod 
\item [{sampler}] a program to monitor state changes of a clockwork system
\item [{filter}] a program to reduce the level of sampled output
\item [{scope}] a program to transform output from sampler to a convenient
oscilloscipe form
\end{description}
For the remainder of this guide, the above programs are split between
those that run clockwork (`Language') and those that provide interfaces
(`Tools').


\subsection{Acknowledgements}

The language and tools making up Latproc are built using a variety
of open source tools and platforms, including;
\begin{itemize}
\item anet tcp wrappers (part of redis - http://redis.io)
\item boost (http://boost.org/) - various c++ bits and pieces
\item IgH EtherCAT Master for Linux (http://www.etherlab.org/en/ethercat/index.php)
\item libmodbus (http://libmodbus.org/) - for communication with modbus/tcp
terminals
\item zeromq (http://www.zeromq.org/) - for inter-program messaging
\item mqtt (http://mqtt.org)
\end{itemize}
We developed the software using open source development tools: GNU
gcc, bison and flex and used Linux and MacOSX workstations.


\section{Language}

The clockwork language is finite state machine based, designed to
monitor various statemachines and to generate events when certain
conditions arise in a machine interfacing and control environment.
The program provides the user with a way to build a model of a machine
and to simulate or change its behaviour. Clockwork is event-based
and inherently parallel and can be used to model many systems that
can be represented using finite state machines.

There are two language drivers at present, \emph{cw} and \emph{iod}.
The difference between them is that iod includes facilities to talk
to I/O hardware via the IgH EtherCAT Master for Linux and cw does
not. We will discuss the use of cw first and discuss the extensions
that iod provides afterwards. Note that both iod and cw are able to
communicate with devices over MQTT with some limitations as the MQTT
implementation is not complete. Through these hardware interfaces,
clockwork can be used to design and build complex control systems
that interface to the physical world.


\subsection{Getting started}

The file: README, that comes with the latproc distribution explains
how to build the cw program. The essense of the process is:
\begin{enumerate}
\item pull the latproc project from git

\begin{lyxcode}
git~clone~git://github.com/latproc/latproc.git~latproc
\end{lyxcode}
\item change to the latproc directory and build the interpreter

\begin{lyxcode}
cd~latproc/iod

make~-f~Makefile.cw~
\end{lyxcode}
\end{enumerate}
The `make' process should produce a file latproc/iod/cw that can be
copied to a convenient location (eg /usr/local/bin)


\subsubsection{Hello World}

Clockwork accepts a list of files on the commandline; it reads all
the state machine descriptions and connections within those files
and then starts a virtual environment that runs those machines. For
the sake of brevity, clockwork refers to finite state machine definitions
simply as 'machines'. A machine requires a \emph{definition}, that
describes the states the machine has and how the machine moves between
those states and at least one \emph{instantiation}, that causes a
machine to be started in the virtual environment.

When an instantiated machine is enabled, it enters its INIT state,
causing an event handler to be executed. Each time a machine changes
state the associated event handler is executed.

Here is a program that will display a message on the terminal:
\begin{lyxcode}
Hello~MACHINE~\{~
\begin{lyxcode}
ENTER~INIT~\{~
\begin{lyxcode}
LOG~\textquotedbl{}Hello~World\textquotedbl{};~

SHUTDOWN
\end{lyxcode}
\}
\end{lyxcode}
\}~

hello~Hello;
\end{lyxcode}
The above program can be executed by saving it into a file (eg hello.cw)
and running cw:
\begin{lyxcode}
cw~hello.cw
\end{lyxcode}
The output should be displayed as:
\begin{lyxcode}
-{}-{}-{}-{}-{}-{}-~hello:~Hello~World~-{}-{}-{}-{}-{}-{}-
\end{lyxcode}
Note that case is important in Clockwork and that the clockwork language
uses uppercase letters for its reserved words and builtin machine
definitions. In Clockwork, instructions can only be executed inside
event handlers, so to log a simple message first requires that we
define a class of state machine and then an event handler within that
state machine. 

The example above defines a class of state machine called 'Hello'
and an event handler for entry to INIT; our LOG statment is executed
when an instance of the 'Hello' class of machine is started.

Note that after the LOG statement, our program has a SHUTDOWN statement;
Clockwork is intended to be used to monitor system states or provide
ongoing control functions so it normally does not exit; the SHUTDOWN
statement tells the driver to shutdown the virtual environment, stopping
all machines.


\subsubsection{Light Sensor}

Here is an example of how Clockwork can be used to define a monitor
that turns a light on or off once there is no activity in a room.
We start with a sensor and a switch, with the idea that when the sensor
comes on there is activity so we turn the light on. When the sensor
goes off, we turn off the light. For the time being, we use the builtin
statemachine called 'FLAG' to simulate the sensor and the switch.
A FLAG has two states, \emph{on} and \emph{off}.

We can generally define things in any order, so lets define our light
controller first.
\begin{lyxcode}
LightController~MACHINE~sensor,~light\_switch~\{

~~~~active~WHEN~sensor~IS~on;

~~~~inactive~DEFAULT;

~~~~ENTER~active~\{~SET~light\_switch~TO~on~\}

~~~~ENTER~inactive~\{~SET~light\_switch~TO~off~\}~

\}
\end{lyxcode}
The definition simply says that when the sensor is on, the light controller
is active and the light should be turned on. Otherwise, the light
controller is inactive and the light should be turned off. The ENTER
methods are executed each time the MACHINE enters a given state. Notice
that we do not initialise the light, when the program starts, the
LightController will determine what to do from the rules we have supplied.

The LightController needs two parameters; a sensor and a light switch.
For the time being, we instantiate our Flags and our controller as
follows:
\begin{lyxcode}
sensor~FLAG;~

switch~FLAG;

controller~LightController~sensor,~switch;
\end{lyxcode}
Note that these entries can be given in any order.


\subsection{Communicating with Clockwork servers}

When a Clockwork program is being run, you can interact with it using
a command interface or via a web page. The simplest method got get
started with is the command interface, the web interface needs some
extra configuration for the web server. Refer to section \ref{sub:iosh}
for further information about iosh.

The latproc source includes some PHP code that provides a simple web
view of the state machines in the executing program. The program requires
PHP version 5.3 and has been tested with apache and with minihttpd.
{[}further explanation of the setup to be done{]}.


\subsection{Soure code conventions and file structure}

When writing programs for Clockwork, program source can be split between
any number of files within a user-nominated directory structure. Files
and directories are provided to cw on the commandline and the program
scans all files in the directories to build a consistent set of definitions
from the fragments found within the files. There is no requirement
to list the files in any particular order but it is an error if a
definition is used but not provided or if a definition is provided
more than once.
\begin{itemize}
\item Program text is freeform, where line breaks, tabs and spaces are all
treated equally. 
\item Comments can be started with `\#' and continue to the end-of-line
or can be started by `/{*}' and ended by `{*}/'.
\item Statements must be separated by semicolon (`;') but the semicolon
before the closing brace (`\}') that ends a group of statements may
still be given.
\end{itemize}

\section{Tools}


\subsection{iosh\label{sub:iosh}}

Clockwork and iod both provide support for a simple shell, called
iosh via the �mq (zeromq) network library. To connect to the clockwork
server, simply run iosh:
\begin{lyxcode}
\$~iosh

Connecting~to~tcp://127.0.0.1:5555

Enter~HELP;~for~help.~Note~that~';'~is~required~at~the~end~of~each~command~~~use~exit;~or~ctrl-D~to~exit~this~program~

>~
\end{lyxcode}
at the prompt, enter any supported command, as follows:
\begin{description}
\item [{DEBUG~machine~on|off}] start/stops debug messages for the device
\item [{DEBUG~debug\_group~on|off}] starts/stops debug messages for all
the devices in the given group
\item [{DISABLE~machine}] disables a machine; in the case of a POINT,
it is turned off, other machines simply sit in the current state and
do not process events or monitor states
\item [{EC~command}] send a command to the ethercat tool (iod only)
\item [{ENABLE~machine\_name}] enable a machine; set the machine state
to its initial state and have it begin processing events and monitoring
states
\item [{GET~machine\_name}] display the state of the names machine
\item [{LIST}] show a list of all machines
\item [{LIST~{[}group\_name{]}}] show a list of machines and their current
state and properties in JSON format, optionally limit the list to
the named group.
\item [{MASTER}] display the ethercat master state (iod only)
\item [{MODBUS~export}] write the modbus export configuration to the export
file (the file name is configured on the commandline when cw or iod
is started
\item [{MODBUS~group~address~new\_value}] simulate a modbus event to
change the given element to the new value
\item [{PROPERTY~machine\_name~property\_name~new\_value}] set the value
of the given property
\item [{QUIT}] exit the program
\item [{RESUME~machine\_name}] enable a machine by reentering the state
it was in when it was disabled.
\item [{SEND~command}] send the event, given in target\_machine\_name
'.' event\_name form.
\item [{SET~machine\_name~TO~state\_name}] attempt to set the named
machine to the given state
\item [{SLAVES}] display information about the known EtherCAT slaves
\item [{TOGGLE~machine\_name}] changes from the on state to off or vice-versa,
only usable on machines with both an on and off state.
\end{description}

\subsection{persistd}


\subsection{beckhoffd}


\subsection{modbusd}


\subsection{device\_connector}


\section{Other Features}


\subsection{Connecting other devices}

Currently external devices can be connected to cw and iod by use of
the EXTERNAL machine class. To define a connection:
\begin{itemize}
\item instantiate an EXTERNAL machine
\item set parameters on that machine:

\begin{description}
\item [{HOST}] a string with the name or ip address of the host. The sepecial
host name `{*}' indicates that the program should use a publisher/subscriber
messaging model and not expect any replies.
\item [{PORT}] a number with the port to connect to on the remote machine
\end{description}
\end{itemize}
When a message is sent to the machine defined in this way, a connection
is made via �MQ and the message is sent. The connection is help open,
ready for more messages.


\section{Examples}


\subsection{Patterns}

The following demonstrates how to detect a message and extract data
from it. In particualr, this machine looks for a four character message
beginning with `c' and ending with `l'. When a match is detected,
it will set a variable called `single' to the first character of the
message and the variable all to the first \$n\$ alphabetic letters.

Note that another machine might send the message using:
\begin{lyxcode}
pattern\_test.message~:=~'curl'
\end{lyxcode}
\begin{lstlisting}
PatternTest MACHINE {
	OPTION message "";
	found WHEN message MATCHES `c..l`;
	not_found DEFAULT; 
	ENTER found {
		single := COPY `[A-Za-z]` FROM message; 		
		all := COPY ALL `[A-Za-z]` FROM message;
	}
}
pattern_test PatternTest;
\end{lstlisting}



\subsection{Latched input}

Here is an example of how to latch an input; once the input comes
on the latched input comes on and stays on until it is reset

\begin{lstlisting}
GenericLatch MACHINE input {
	on STATE;
	off INITIAL;
	RECEIVE input.on_enter { SET SELF TO on }
	TRANSITION on TO off ON reset;
}
\end{lstlisting}



\subsection{Calling functions}

Machines can receive messages from other machines. To send a message
`start` to a machine called `other', the statement:
\begin{lyxcode}
SEND~start~TO~other;
\end{lyxcode}
would be used. Sometimes, we refer to these messages as `commands'
and the two terms can be used interchangeably. The following example
demonstrates the `CALL' syntax to send the message. Apart from the
name difference, a CALL statement will block until the command handled
at the receiving machine.

\begin{lstlisting}
CommandTest MACHINE other {
	a DEFAULT;
	ENTER a { 
		LOG "a"; 
		CALL x ON SELF;
		CALL y ON other; 
	}
	COMMAND x { 
		LOG "x on CommandTest called" }
	}

OtherTest MACHINE {
	COMMAND y { LOG "y on OtherTest called" }
}
o OtherTest;
test CommandTest o;
\end{lstlisting}



\subsection{Dining philosopers}

The following simulates several philosophers dining at a circular
table. There are seven philosophers and only seven chopsticks so chopsticks
must be shared as two chopsticks are required to eat. The program
uses resource locking to guarantee that chopsticks are not being used
by different philosplers simultaneously.

\begin{lstlisting}
Chopstick MACHINE {
	OPTION tab Test;
	OPTION owner "noone";
	free STATE;
	busy STATE;
}
Philosopher MACHINE left, right {
	OPTION tab Test;
    OPTION eat_time 20;
    OPTION timer 20;
        full FLAG;
    okToStart FLAG;
    okToStop FLAG;
    finished WHEN SELF IS finishing || SELF IS finished && TIMER < timer;
	finishing WHEN left.owner == SELF.NAME && right.owner == SELF.NAME  
		&& TIMER >= eat_time; 
	eating WHEN left.owner == SELF.NAME && right.owner == SELF.NAME,
        TAG full WHEN TIMER > 10;
	starting WHEN left.owner == "noone" && right.owner == "noone";
	waiting DEFAULT;
    ENTER INIT {
        eat_time := (NOW % 10) * 10;
        SET okToStart TO on;
        SET okToStop TO on;
    }
	ENTER starting {
		LOCK left;
		IF (left.owner == "noone") {
            LOG "got left";
			left.owner := SELF.NAME;
			LOCK right;
			IF (right.owner == "noone") {
				right.owner := SELF.NAME;
                LOG "got right";
			}
			ELSE {
				UNLOCK left;
				UNLOCK right;
			}
		}
		ELSE {
			UNLOCK left;
		}
	}
	ENTER finished {
        LOG "finished"; 
		left.owner := "noone";
		right.owner := "noone";
		UNLOCK right;
		UNLOCK left;
		timer := 10 * (TIMER + 1);
	}
    TRANSITION INIT TO starting REQUIRES okToStart IS on;
    TRANSITION eating TO finished REQUIRES okToStop IS on; }

c01 Chopstick; c02 Chopstick; c03 Chopstick; 
c04 Chopstick; c05 Chopstick; c06 Chopstick; c07 Chopstick;
phil1 Philosopher c01, c02; 
phil2 Philosopher c02, c03; 
phil3 Philosopher c03, c04; 
phil4 Philosopher c04, c05; 
phil5 Philosopher c05, c06; 
phil6 Philosopher c06, c07; 
phil7 Philosopher c07, c01; 
\end{lstlisting}

\end{document}
